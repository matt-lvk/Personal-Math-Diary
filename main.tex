\documentclass[12pt]{article}

\usepackage[utf8]{inputenc}

\usepackage[left=1.5cm,right=1.5cm,top=3cm,bottom=3cm]{geometry}
\geometry{verbose}

%\usepackage{newtxtext}
\usepackage{enumitem}
\setenumerate[1]{leftmargin=3em}

\usepackage{amssymb}
\usepackage{amsmath}
\usepackage{amsthm}
\usepackage{amsopn}

\usepackage{mathrsfs}
\usepackage{mdframed}
\usepackage{tikz-cd}

\usepackage{esint}

\usepackage{thmtools, thm-restate}
\usepackage{hyperref}

\usepackage{verbatim}
\usepackage{mathtools}
\usepackage{fullpage}
\usepackage{dsfont}

\usepackage{physics}
\usepackage{setspace}
\usepackage{epstopdf}
\usepackage[T1]{fontenc}
\usepackage{graphicx}

\usepackage{scalerel,stackengine}
\usepackage{float}
\usepackage{ulem}

\usepackage{authblk}

\usepackage{soul}

\usepackage[sort&compress]{natbib}





%%%%%%%%%% Mathmematical Things %%%%%%%%%%

%%%%% Theorem, Cor, Lem, etc. %%%%%

\theoremstyle{remark}
\newtheorem{Remark}{Remark}[section]

\theoremstyle{plain}
\newtheorem{Theorem}{Theorem}[section]
\newtheorem{Lemma}{Lemma}[section]
\newtheorem{Corollary}{Corollary}[section]
\newtheorem{Definition}{Definition}[section]
\newtheorem{Proposition}{Proposition}[section]

\newtheorem{assump}{Assumption}
\newtheorem{assumpB}{Assumption}
\renewcommand\theassump{A\arabic{assump}}
\renewcommand\theassumpB{B\arabic{assumpB}}
\DeclareRobustCommand{\rchi}{{\mathpalette\irchi\relax}}
\newcommand{\irchi}[2]{\raisebox{\depth}{$#1\chi$}} % inner command, used by \rchi


\newtheorem{assumptionex}{Assumption}
\newenvironment{assumption}
{\pushQED{\qed}\renewcommand{\qedsymbol}{$\bullet$}\assumptionex}
{\popQED\endassumptionex}

\setcounter{secnumdepth}{2}
\setcounter{tocdepth}{2}
%\addcontentsline{toc}{section}{Abstract}

%%%%% New operators %%%%%

\DeclareMathOperator{\sign}{sign}
\DeclareMathOperator{\erfc}{erfc}
\DeclareMathOperator{\diverg}{div}
\DeclareMathOperator{\Op}{Op}
\DeclareMathOperator{\Ima}{Im}
\DeclareMathOperator{\RePart}{Re}
\DeclareMathOperator{\diag}{diag}
\DeclareMathOperator{\supp}{supp}

\def \qed {\hfill \vrule height6pt width 6pt depth 0pt}
\newcommand{\al}{\alpha}
\newcommand{\R}{\mathbb{R}}
\newcommand{\N}{\mathbb{N}}
\newcommand{\Z}{\mathbb{Z}}
\newcommand{\C}{\mathbb{C}}
\newcommand{\Q}{\mathbb{Q}}
\newcommand{\p}{\partial}
\newcommand{\unity}{{1\!\!\!\:\mathrm{l}}}
\newcommand{\rightv}{\overset{\rightharpoonup}}
\newcommand{\harpoon}{\overset{\rightharpoonup}}
\newcommand{\hsnorm}[1]{\norm{#1}_{\text{HS}}}
\newcommand{\opnorm}[1]{\norm{#1}_{\text{op}}}
\newcommand{\trnorm}[1]{\norm{#1}_{\Tr}}

\usepackage{ulem}

\usepackage{natbib} % reference manager
%\usepackage[normalem]{ulem} %to strike the words

%%%%% Displaybreaks %%%%%
\allowdisplaybreaks

%%%%% Equation numbering %%%%%
\numberwithin{equation}{section}


\stackMath
\newcommand\reallywidehat[1]{%
	\savestack{\tmpbox}{\stretchto{%
			\scaleto{%
				\scalerel*[\widthof{\ensuremath{#1}}]{\kern-.6pt\bigwedge\kern-.6pt}%
				{\rule[-\textheight/2]{1ex}{\textheight}}%WIDTH-LIMITED BIG WEDGE
			}{\textheight}% 
		}{0.5ex}}%
	\stackon[1pt]{#1}{\tmpbox}%
}


%\renewcommand{\geqslant}{\geq}
%\renewcommand{\leqslant}{\leq}
\renewcommand{\tilde}{\widetilde}
\renewcommand{\leq}{\leqslant}
\renewcommand{\geq}{\geqslant}
\DeclarePairedDelimiter\ceil{\lceil}{\rceil}

\title{Writing Husimi Measure under Hartree Fock equation}
\date{}
\author{}

\begin{document}
	
\section{May 2023}

\subsection{Safety Stock}
\begin{Theorem}
	Let the lead-time $T$ be normally distributed, i.e. $T\sim N(\mu_\ell, \sigma^2_\ell)$ and denote the set of demand of period $i$ as $\{D_i\}_{i=1}^T$ where $D_i \sim N(\mu_d, \sigma^2_d)$ for each $1\leq i \leq T$. Then the safety stock level, $SS$ satisifies the following equation
		\[
		SS = z_\alpha \sqrt{\sigma_d^2 \mu_\ell + \mu_d^2 \sigma^2_\ell},
		\]
		where $z_\alpha$ is the Z-value of a desired $\alpha$ which is chosen.
\end{Theorem}

\begin{proof}
	Denote the demand between within lead-time as $D(T) := \sum_{i=1}^N D_i$. The safety stock level is then set at
		\[
		SS = z_\alpha \sqrt{Var(D(T))}.
		\]
	Thus we need only to prove $Var(D(T))$. 
	
	From Towering-property, note that 
	\begin{align*}
		\mathbb{E}[D(T)] & = \mathbb{E} \big[\mathbb{E}[D(T)|T]\big]\\
		&
		= \mathbb{E}\big[T \mu_d\big] = \mu_d \mathbb{E}[T] = \mu_d \mu_\ell.
	\end{align*}
	Furthemore, again with Towering-property, we have
	\begin{align*}
		\mathbb{E}[D^2(T)] & = \mathbb{E} \big[\mathbb{E}[D^2(T)|T]\big]
			\\
			& = \mathbb{E}\big[Var[D(T)|T] + \big(\mathbb{E}[D(T)|T]\big)^2\big]
			\\			
			& = \mathbb{E}\big[T \sigma_d^2 + \big(T \mu_d \big)^2\big]
			\\
			& = \sigma_d^2 \mathbb{E}[T] + \mu_d^2\mathbb{E}[T^2]
			\\
			& =  \sigma_d^2 \mu_\ell + \mu_d^2\left(Var(T) + (\mathbb{E}[T])^2\right)
			\\
			& = \sigma_d^2 \mu_\ell + \mu_d^2\left(\sigma^2_\ell+ \mu_\ell^2\right).
	\end{align*}
Finally,
	\begin{align*}
		Var(D(T)) &= \mathbb{E}[D^2(T)] - \big(\mathbb{E}[D(T)]\big)^2
		\\
		& =  \sigma_d^2 \mu_\ell + \mu_d^2\left(\sigma^2_\ell+ \mu_\ell^2\right) - (\mu_d \mu_\ell)^2
		\\
		& = \sigma_d^2 \mu_\ell + \mu_d^2 \sigma^2_\ell.
	\end{align*}
\end{proof}
\clearpage

\subsection{Binomal approx to Poisson}
\begin{Theorem}
	For $\lambda = np $,
	$$ \lim_{n  \to \infty} \binom{n}{k} p^k {1 - p}^{n - k} = \frac {\lambda^k} {k!} e^{-\lambda}$$
\end{Theorem}



\end{document}
